\documentclass[a4paper]{article}
% Kodowanie latain 2
%\usepackage[latin2]{inputenc}
\usepackage[T1]{fontenc}
% Można też użyć UTF-8
\usepackage[utf8]{inputenc}

% Język
\usepackage[polish]{babel}
% \usepackage[english]{babel}

% Rózne przydatne paczki:
% - znaczki matematyczne
\usepackage{amsmath, amsfonts}
% - wcięcie na początku pierwszego akapitu
\usepackage{indentfirst}
% - komenda \url 
\usepackage{hyperref}
% - dołączanie obrazków
\usepackage{graphics}
% - szersza strona
\usepackage[nofoot,hdivide={2cm,*,2cm},vdivide={2cm,*,2cm}]{geometry}
\frenchspacing
% - brak numerów stron
\pagestyle{empty}

% dane autora
\author{Martyna Sowa}
\title{Sprawdzian PWI}
\date{\today}

% początek dokumentu
\begin{document}
\section{Zadanie 1}
${\rho \frac{D\textbf{u}}{Dt} = \rho(\frac{\partial\textbf{u}}{\partial t } + \textbf{u} \cdot \bigtriangledown \textbf{u}) = - \bigtriangledown \overline{p} + \bigtriangledown \cdot \{ \mu (\bigtriangledown \textbf{u} + (\bigtriangledown \textbf{u} )^T - \frac{2}{3} (\bigtriangledown \cdot \textbf{u})\textbf{I} ) \} + \rho \textbf{g} }$

\bigskip

${\tilde{f}(\xi)=\int_{-\infty}^{\infty} f(x) e^{-2 \pi i x \xi} dx}$

\bigskip

${\mathbb{P} (\hat{X}_n - z_{1-\frac{\alpha}{2}} \frac{\sigma}{\sqrt{n}} \leq \mathbb{E}X \leq \hat{X}_n + z_{1-\frac{\alpha}{2}} \frac{\sigma}{\sqrt{n}}) \approx 1 - \alpha }$

\bigskip

${
\begin{bmatrix}
1 & 2 \\
3 & 4
\end{bmatrix}
\otimes
\begin{bmatrix}
0 & 5 \\
6 & 7
\end{bmatrix}
=
\begin{bmatrix}
1
\begin{bmatrix}
0 & 5 \\
6 & 7
\end{bmatrix}
&
2
\begin{bmatrix}
0 & 5 \\
6 & 7
\end{bmatrix}
\\
3
\begin{bmatrix}
0 & 5 \\
6 & 7
\end{bmatrix}
& 
4
\begin{bmatrix}
0 & 5 \\
6 & 7
\end{bmatrix}
\end{bmatrix}
=
\begin{bmatrix}
0 & 5 & 0 & 10 \\
6 & 7 & 12 & 14 \\
0 & 15 & 0 & 20 \\
18 & 21 & 24 & 28
\end{bmatrix}
}$

\section{Zadanie 2}
1. ssh-keygen

2. ssh-copy-id -i klucz\_do\_serwera.pub s324251@pwi.ii.uni.wroc.pl

3.

echo ''\# PWI-sprawdzian-s324'' >> README.md

git init

git add README.md

cp ../klucz\_do\_repozytorium.pub .

git add klucz\_do\_repozytorium.pub

git commit -m "first commit"

git branch -M main

git remote add origin https://github.com/martiesova/PWI-sprawdzian-s324.git

git push -u origin main

4.

nano ~/.ssh/config

Zawartość pliku:

Host pwi-sprawdzian

\quad User s324251
        
\quad HostName pwi.ii.uni.wroc.pl

5. Nie rozumiem zadania

\section{Zadanie 3}
1. git clone https://github.com/martiesova/PWI-sprawdzian-s324

Nie wiem dlaczego rozwiązanie z generowaniem klucza jest brzydkie. Nie potrzebowałam też generować klucza na serwerze i nie wiem dlaczego miało by to być konieczne.

2.

wget http://www.ii.uni.wroc.pl/~lisu/zadanie.tar.gz

tar -xf ./zadanie.tar.gz

git add zadanie/*

git commit

3.

echo -n s324251 | md5sum

Nie udało mi się znaleźć folderu o nazwie odpowiadającej wynikowi

4.

scp sprawdzian.tex  s324251@pwi.ii.uni.wroc.pl:~/PWI-sprawdzian-s324

scp sprawdzian.pdf  s324251@pwi.ii.uni.wroc.pl:~/PWI-sprawdzian-s324

git add sprawdzian.tex

git add sprawdzian.pdf

git commit

git push origin master

\end{document}